\documentclass[11pt,a4paper]{article}
\usepackage{isabelle,isabellesym}

% further packages required for unusual symbols (see also
% isabellesym.sty), use only when needed

\usepackage{amssymb}
  %for \<leadsto>, \<box>, \<diamond>, \<sqsupset>, \<mho>, \<Join>,
  %\<lhd>, \<lesssim>, \<greatersim>, \<lessapprox>, \<greaterapprox>,
  %\<triangleq>, \<yen>, \<lozenge>
\usepackage{wasysym}
% for \<hole>

% this should be the last package used
\usepackage{pdfsetup}

% urls in roman style, theory text in math-similar italics
\urlstyle{rm}
\isabellestyle{it}

% for uniform font size
\renewcommand{\isastyle}{\isastyleminor}

\renewcommand{\isamarkupchapter}[1]{\section{#1}}
\renewcommand{\isamarkupsection}[1]{\subsection{#1}}
\renewcommand{\isamarkupsubsection}[1]{\subsubsection{#1}}
\renewcommand{\isamarkupsubsubsection}[1]{\paragraph{#1}}

\begin{document}

\title{Timed Automata}
\author{Simon Wimmer}

\maketitle
\begin{abstract}
  Timed automata are a widely used formalism for modeling real-time systems, which is employed
  in a class of successful model checkers such as UPPAAL \cite{Larsen1997},
  HyTech \cite{Henzinger97hytech} or Kronos \cite{Kronos97}.
  This work formalizes the theory for the subclass of diagonal-free timed automata, which is
  sufficient to model many interesting problems.
  We first define the basic concepts and semantics of diagonal-free timed automata.
  Based on this, we prove two types of decidability results for the language emptiness problem.

  The first is the classic result of Alur and Dill \cite{alur_automata_1990,alur_theory_1994},
  which uses a finite partitioning of the state space into so-called \textit{regions}.

  Our second result focuses on an approach based on \textit{Difference Bound Matrices (DBMs)},
  which is practically used by model checkers.
  We prove the correctness of the basic forward analysis operations on DBMs.
  One of these operations is the Floyd-Warshall algorithm for the all-pairs
  shortest paths problem.
  To obtain a finite
  search space, a widening operation has to be used for this kind of analysis.
  We use Patricia Bouyer's \cite{Bou_Forward_Analysis} approach to prove that this widening operation
  is correct in the sense that DBM-based forward analysis in combination with the widening operation
  also decides language emptiness. The interesting property of this proof is that the first
  decidability result is reused to obtain the second one.
\end{abstract}

\setcounter{tocdepth}{2}
\tableofcontents
\newpage

% sane default for proof documents
\parindent 0pt\parskip 0.5ex

% generated text of all theories
\input{session}

% optional bibliography
\bibliographystyle{alpha}
\bibliography{root}

\end{document}

%%% Local Variables:
%%% mode: latex
%%% TeX-master: t
%%% End:
